\documentclass[10pt]{article}
\usepackage[margin=1in]{geometry}
\usepackage{amsmath,amssymb,graphicx,hyperref,mathrsfs,siunitx}
\usepackage[small,bf]{caption}
\title{Isothermobaric Ternary Phase Diagrams}
\author{Trevor Keller}
\date{\small \today}

\frenchspacing
\hypersetup{linktocpage, colorlinks, citecolor=black, filecolor=black, linkcolor=black, urlcolor=blue}

\begin{document}
\maketitle

It is always useful to view the thermodynamic phase diagram as a sanity check on
kinetic simulations. This is commonly done for binary alloy systems. The process
for generating a correct ternary phase diagram is less routine; this document
summarizes the derivation from the system free energy functional as a
constrained minimization problem.

An exposition of the method is given in part on
\href{https://math.stackexchange.com/questions/632/validating-a-mathematical-model-lagrange-formulation-and-geometry}
{math.stackexchange.com}. The Python source code in this repository provides a
complete implementation.

\section*{Binary Two-Phase}

Consider a binary system with two phases, the free energies of which are defined
by the functions $G_{\alpha}(x)$ and $G_{\beta}(x)$. The system will have volume
fraction $v_{\alpha}$ of phase $\alpha$ and $v_{\beta}$ of phase $\beta$. We
seek to determine the equilibrium phase compositions ($x_{\alpha}$, $x_{\beta}$)
and volume fractions that minimize the free energy of the total system, subject
to two constraints. To do so, we'll make use of Lagrange multipliers.

\begin{align}
  G(x_{\alpha}, x_{\beta}) &= v_{\alpha} G_{\alpha}(x_{\alpha})
                            + v_{\beta} G_{\beta}(x_{\beta})\\
  v_{\alpha} x_{\alpha} + v_{\beta} x_{\beta} &= x\\
  v_{\alpha} + v_{\beta} &= 1
\end{align}
where $x$ is the system composition. The system functional and integrand are

\begin{align}
  \mathcal{L}[x_{\alpha}, x_{\beta}, v_{\alpha}, v_{\beta}, \lambda_1, \lambda_2]
  &= \int\limits_VF\mathrm{d}V\\
  F(x_{\alpha}, x_{\beta}, v_{\alpha}, v_{\beta}, \lambda_1, \lambda_2)
  &= v_{\alpha} G_{\alpha}(x_{\alpha}) + v_{\beta} G_{\beta}(x_{\beta})
    - \lambda_1(v_{\alpha} x_{\alpha} + v_{\beta} x_{\beta} - x)
    - \lambda_2(v_{\alpha} + v_{\beta} - 1).
\end{align}
Therefore, the system of Euler-Lagrange Equations is

\begin{align}
  \label{eqn:el1}
  \frac{\delta\mathcal{L}}{\delta x_{\alpha}} = \frac{\partial F}{\partial x_{\alpha}} = 0
  &= v_{\alpha} \frac{\partial G_{\alpha}}{\partial x_{\alpha}} - \lambda_1 v_{\alpha}\\
  \label{eqn:el2}
  \frac{\delta\mathcal{L}}{\delta x_{\beta}} = \frac{\partial F}{\partial x_{\beta}} = 0
  &= v_{\beta} \frac{\partial G_{\beta}}{\partial x_{\beta}} - \lambda_1 v_{\beta}\\
  \label{eqn:el3}
  \frac{\delta\mathcal{L}}{\delta v_{\alpha}} = \frac{\partial F}{\partial v_{\alpha}} = 0
  &= G_{\alpha}(x_{\alpha}) - \lambda_1x_{\alpha} - \lambda_2\\
  \label{eqn:el4}
  \frac{\delta\mathcal{L}}{\delta v_{\beta}} = \frac{\partial F}{\partial v_{\beta}} = 0
  &= G_{\beta}(x_{\beta}) - \lambda_1x_{\beta} - \lambda_2
\end{align}
Solving Eqns.~\ref{eqn:el1}~\&~\ref{eqn:el2} for $\lambda_1$ gives

\begin{align}
  \lambda_1 &= \frac{\partial G_{\alpha}}{\partial x_{\alpha}}
             = \frac{\partial G_{\beta}}{\partial x_{\beta}}\\
  \label{eqn:chempot}
  \therefore \frac{\partial G_{\alpha}}{\partial x_{\alpha}} &=
             \frac{\partial G_{\beta}}{\partial x_{\beta}}.
\end{align}
Solving Eqns.~\ref{eqn:el3}$~\&~$\ref{eqn:el4} for $\lambda_2$ and substituting
Eqn.~\ref{eqn:chempot} gives

\begin{align}
  \lambda_2 &= G_{\alpha}(x_{\alpha})
             - \frac{\partial G_{\alpha}}{\partial x_{\alpha}} x_{\alpha}
             = G_{\beta}(x_{\beta})
             - \frac{\partial G_{\alpha}}{\partial x_{\alpha}} x_{\beta}\\
  \label{eqn:grandpot}
  \therefore G_{\beta}(x_{\beta}) &= G_{\alpha}(x_{\alpha})
                                   - \frac{\partial G_{\alpha}}{\partial x_{\alpha}}
                                     \left(x_{\alpha} - x_{\beta}\right).
\end{align}
Eqn.~\ref{eqn:chempot} represents equality of chemical potential, while
Eqn.~\ref{eqn:grandpot} represents equality of grand potential energy. This pair
of equations can be solved to determine the two unknown compositions needed to
specify the common tangent line containing points $G_{\alpha}(x_{\alpha})$,
$G(x)$, and $G_{\beta}(x_{\beta})$.

\newpage
\section*{Ternary Two-Phase}

Consider a ternary system with two phases, the free energies of which are
defined by the functions $G_{\alpha}(x_1, x_2)$ and $G_{\beta}(x_1, x_2)$. The
system will have volume fraction $v_{\alpha}$ of phase $\alpha$ and $v_{\beta}$
of phase $\beta$. We seek to determine the equilibrium phase compositions
($x_1^{\alpha}, x_2^{\alpha}, x_1^{\beta}, x_2^{\beta}$) and volume fractions
that minimize the free energy of the total system, subject to three constraints.
Unlike the binary case, we must make use of the constraints directly by
including Euler-Lagrange equations with respect to the Lagrange multipliers.
This is because the common tangent \emph{line} between the two points does not
completely specify the equilibrium tangent \emph{plane}.

\begin{align}
  G(x_1^{\alpha}, x_2^{\alpha}, x_1^{\beta}, x_2^{\beta})
  &= v_{\alpha} G_{\alpha}(x_1^{\alpha},x_2^{\alpha})
   + v_{\beta} G_{\beta}(x_1^{\beta}, x_2^{\beta})\\
  v_{\alpha} x_1^{\alpha} + v_{\beta} x_1^{\beta} &= x_1\\
  v_{\alpha} x_2^{\alpha} + v_{\beta} x_2^{\beta} &= x_2\\
  v_{\alpha} + v_{\beta} &= 1
\end{align}
where A and B label phases, 1 and 2 label components, $x_1$ and $x_2$ are the
system compositions, and compositions with labels $x_i^j$ represent the
composition of component $i$ in phase $j$. The integrand is

\begin{align}
  \nonumber
  F(x_1^{\alpha}, x_2^{\alpha}, x_1^{\beta}, x_2^{\beta})
  &= v_{\alpha} G_{\alpha}(x_1^{\alpha}, x_2^{\alpha})
   + v_{\beta} G_{\beta}(x_1^{\beta}, x_2^{\beta})\\
  \nonumber
  &- \lambda_1(v_{\alpha} x_1^{\alpha} + v_{\beta} x_1^{\beta} - x_1)\\
  \nonumber
  &- \lambda_2(v_{\alpha} x_2^{\alpha} + v_{\beta} x_2^{\beta} - x_2)\\
  &- \lambda_3(v_{\alpha} + v_{\beta} - 1)
\end{align}
The system of Euler-Lagrange Equations becomes

\begin{align}
  \label{eqn:2el1}
  \frac{\partial F}{\partial x_1^{\alpha}} = 0
  &= v_{\alpha} \frac{\partial G_{\alpha}}{\partial x_1^{\alpha}}
   - \lambda_1 v_{\alpha}\\
  \label{eqn:2el2}
  \frac{\partial F}{\partial x_1^{\beta}} = 0
  &= v_{\beta} \frac{\partial G_{\beta}}{\partial x_1^{\beta}}
   - \lambda_1 v_{\beta}\\
  \label{eqn:2el3}
  \frac{\partial F}{\partial x_2^{\alpha}} = 0
  &= v_{\alpha} \frac{\partial G_{\alpha}}{\partial x_2^{\alpha}}
   - \lambda_2 v_{\alpha}\\
  \label{eqn:2el4}
  \frac{\partial F}{\partial x_2^{\beta}} = 0
  &= v_{\beta} \frac{\partial G_{\beta}}{\partial x_2^{\beta}}
   - \lambda_2 v_{\beta}\\
  \label{eqn:2el5}
  \frac{\partial F}{\partial v_{\alpha}} = 0
  &= G_{\alpha} - \lambda_1 x_1^{\alpha} - \lambda_2 x_2^{\alpha}
   - \lambda_3\\
  \label{eqn:2el6}
  \frac{\partial F}{\partial v_{\beta}} = 0
  &= G_{\beta} - \lambda_1 x_1^{\beta} - \lambda_2 x_2^{\beta}
   - \lambda_3\\
  \label{eqn:2el7}
  \frac{\partial F}{\partial \lambda_1} = 0
  &= v_{\alpha} x_1^{\alpha} + v_{\beta} x_1^{\beta} - x_1\\
  \label{eqn:2el8}
  \frac{\partial F}{\partial \lambda_2} = 0
  &= v_{\alpha} x_2^{\alpha} + v_{\beta} x_2^{\beta} - x_2\\
  \label{eqn:2el9}
  \frac{\partial F}{\partial \lambda_3} = 0
  &= v_{\alpha}  + v_{\beta} - 1
\end{align}
Solving Eqns.~\ref{eqn:2el1}~\&~\ref{eqn:2el2} for $\lambda_1$ gives

\begin{align}
  \lambda_1 &= \frac{\partial G_{\alpha}}{\partial x_1^{\alpha}}
             = \frac{\partial G_{\beta}}{\partial x_1^{\beta}}\\
  \label{eqn:chempot1}
  \therefore \frac{\partial G_{\alpha}}{\partial x_1^{\alpha}}
          &= \frac{\partial G_{\beta}}{\partial x_1^{\beta}}
\end{align}
This represents equality of chemical potential of component 1.
Likewise, solving Eqns.~\ref{eqn:2el3}~\&~\ref{eqn:2el4} for $\lambda_2$ gives

\begin{align}
  \lambda_2 &= \frac{\partial G_{\alpha}}{\partial x_2^{\alpha}}
             = \frac{\partial G_{\beta}}{\partial x_2^{\beta}}\\
  \label{eqn:chempot2}
  \therefore \frac{\partial G_{\alpha}}{\partial x_2^{\alpha}}
          &= \frac{\partial G_{\beta}}{\partial x_2^{\beta}}
\end{align}
This represents equality of chemical potential of component 2. Solving
Eqns.~\ref{eqn:2el5}$~\&~$\ref{eqn:2el6} for $\lambda_3$, after substituting
Eqns.~\ref{eqn:chempot1}~\&~\ref{eqn:chempot2}, gives

\begin{align}
  \nonumber
  \lambda_3 &= G_{\alpha}
             - \frac{\partial G_{\alpha}}{\partial x_1^{\alpha}} x_1^{\alpha}
             - \frac{\partial G_{\alpha}}{\partial x_2^{\alpha}} x_2^{\alpha}\\
             &= G_{\beta}
             - \frac{\partial G_{\alpha}}{\partial x_1^{\alpha}} x_1^{\beta}
             - \frac{\partial G_{\alpha}}{\partial x_2^{\alpha}} x_2^{\beta}\\
  \label{eqn:grandpot1}
  \therefore
  G_{\beta} &= G_{\alpha}
             - \frac{\partial G_{\alpha}}{\partial x_1^{\alpha}}(x_1^{\alpha} - x_1^{\beta})
             - \frac{\partial G_{\alpha}}{\partial x_2^{\alpha}}(x_2^{\alpha} - x_2^{\beta}).
\end{align}
This represents equality of the grand potential of phases A and B.

Substituting Eqn.~\ref{eqn:2el9} into Eqns.~\ref{eqn:2el7}~\&~\ref{eqn:2el8}
gives

\begin{align}
  x_1 &= v_{\alpha} x_1^{\alpha} + (1 - v_{\alpha}) x_1^{\beta}\\
  x_2 &= v_{\alpha} x_2^{\alpha} + (1 - v_{\alpha}) x_2^{\beta}
\end{align}
Solving for $v_{\alpha}$ gives us the ternary Lever Rule:
\begin{align}
  \nonumber
  v_{\alpha} &= \frac{x_1 - x_1^{\beta}}{x_1^{\alpha} - x_1^{\beta}}\\
             &= \frac{x_2 - x_2^{\beta}}{x_2^{\alpha} - x_2^{\beta}}\\
  \label{eqn:lever}
  \therefore \frac{x_1 - x_1^{\beta}}{x_1^{\alpha} - x_1^{\beta}}
          &= \frac{x_2 - x_2^{\beta}}{x_2^{\alpha} - x_2^{\beta}}.
\end{align}

The system of Eqns.~\ref{eqn:chempot1}, \ref{eqn:chempot2}, \ref{eqn:grandpot1}
\& \ref{eqn:lever} can be solved to determine the two unknown composition pairs
needed to specify the equilibrium tie line containing the points
$G_{\alpha}(x_1^{\alpha},x_2^{\alpha})$, $G(x_1,x_2)$, and
$G_{\beta}(x_1^{\beta},x_2^{\beta})$.

\newpage
\section*{Ternary Three-Phase}

Consider a ternary system with three phases, the free energies of which are
defined by the functions $G_{\alpha}(x_1, x_2)$, $G_{\beta}(x_1, x_2)$, and
$G_{\gamma}(x_1, x_2)$. The system will have volume fraction $v_{\alpha}$ of
phase $\alpha$, $v_{\beta}$ of phase $\beta$, and $v_{\gamma}$ of phase
$\gamma$. We seek to determine the equilibrium phase compositions
($x_1^{\alpha}, x_2^{\alpha}, x_1^{\beta}, x_2^{\beta}, x_1^{\gamma},
x_2^{\gamma}$) and volume fractions that minimize the free energy of the total
system, subject to three constraints. Like the binary case, we do not make use
of the constraints directly because three points completely specify the common
tangent \emph{plane}.

\begin{align}
  G(x_1^{\alpha}, x_1^{\beta}, x_1^{\gamma}, x_2^{\alpha}, x_2^{\beta}, x_2^{\gamma})
  &= v_{\alpha} G_{\alpha}(x_1^{\alpha},x_2^{\alpha})
    + v_{\beta} G_{\beta}(x_1^{\beta}, x_2^{\beta})
    + v_{\gamma} G_{\gamma}(x_1^{\gamma}, x_2^{\gamma})\\
  v_{\alpha} x_1^{\alpha} + v_{\beta} x_1^{\beta} + v_{\gamma} x_1^{\gamma} &= x_1\\
  v_{\alpha} x_2^{\alpha} + v_{\beta} x_2^{\beta} + v_{\gamma} x_2^{\gamma} &= x_2\\
  v_{\alpha} + v_{\beta} + v_{\gamma} &= 1
\end{align}
where $x_1$ and $x_2$ are the system compositions. The integrand is

\begin{align}
  \nonumber
  F(x_1^{\alpha}, x_1^{\beta}, x_1^{\gamma}, x_2^{\alpha}, x_2^{\beta}, x_2^{\gamma})
  &= v_{\alpha} G_{\alpha}(x_1^{\alpha}, x_2^{\alpha})
   + v_{\beta} G_{\beta}(x_1^{\beta}, x_2^{\beta})
   + v_{\beta} G_{\gamma}(x_1^{\gamma}, x_2^{\gamma})\\
  \nonumber
  &- \lambda_1(v_{\alpha} x_1^{\alpha} + v_{\beta} x_1^{\beta} + v_{\gamma} x_1^{\gamma} - x_1)\\
  \nonumber
  &- \lambda_2(v_{\alpha} x_2^{\alpha} + v_{\beta} x_2^{\beta} + v_{\gamma} x_2^{\gamma} - x_2)\\
  &- \lambda_3(v_{\alpha} + v_{\beta} + v_{\gamma} - 1)
\end{align}
The system of Euler-Lagrange Equations becomes

\begin{align}
  \label{eqn:3el1}
  \frac{\partial F}{\partial x_1^{\alpha}} = 0
  &= v_{\alpha} \frac{\partial G_{\alpha}}{\partial x_1^{\alpha}} - \lambda_1 v_{\alpha}\\
  \label{eqn:3el2}
  \frac{\partial F}{\partial x_1^{\beta}} = 0
  &= v_{\beta} \frac{\partial G_{\beta}}{\partial x_1^{\beta}} - \lambda_1 v_{\beta}\\
  \label{eqn:3el3}
  \frac{\partial F}{\partial x_1^{\gamma}} = 0
  &= v_{\gamma} \frac{\partial G_{\gamma}}{\partial x_1^{\gamma}} - \lambda_1 v_{\gamma}\\
  \label{eqn:3el4}
  \frac{\partial F}{\partial x_2^{\alpha}} = 0
  &= v_{\alpha} \frac{\partial G_{\alpha}}{\partial x_2^{\alpha}} - \lambda_2 v_{\alpha}\\
  \label{eqn:3el5}
  \frac{\partial F}{\partial x_2^{\beta}} = 0
  &= v_{\beta} \frac{\partial G_{\beta}}{\partial x_2^{\beta}} - \lambda_2 v_{\beta}\\
  \label{eqn:3el6}
  \frac{\partial F}{\partial x_2^{\gamma}} = 0
  &= v_{\gamma} \frac{\partial G_{\gamma}}{\partial x_2^{\gamma}} - \lambda_2 v_{\gamma}\\
  \label{eqn:3el7}
  \frac{\partial F}{\partial v_{\alpha}} = 0
  &= G_{\alpha} - \lambda_1 x_1^{\alpha} - \lambda_2 x_2^{\alpha} - \lambda_3\\
  \label{eqn:3el8}
  \frac{\partial F}{\partial v_{\beta}} = 0
  &= G_{\beta} - \lambda_1 x_1^{\beta} - \lambda_2 x_2^{\beta} - \lambda_3\\
  \label{eqn:3el9}
  \frac{\partial F}{\partial v_{\gamma}} = 0
  &= G_{\gamma} - \lambda_1 x_1^{\gamma} - \lambda_2 x_2^{\gamma} - \lambda_3
\end{align}
Solving Eqns.~\ref{eqn:3el1},~\ref{eqn:3el2}~\&~\ref{eqn:3el3}
for $\lambda_1$ gives

\begin{align}
  \lambda_1 &= \frac{\partial G_{\alpha}}{\partial x_1^{\alpha}}
             = \frac{\partial G_{\beta}}{\partial x_1^{\beta}}
             = \frac{\partial G_{\gamma}}{\partial x_1^{\gamma}}\\
  \label{eqn:3chempot1a}
  \therefore \frac{\partial G_{\alpha}}{\partial x_1^{\alpha}}
          &= \frac{\partial G_{\beta}}{\partial x_1^{\beta}}\\
  \label{eqn:3chempot1b}
             \frac{\partial G_{\alpha}}{\partial x_1^{\alpha}}
          &= \frac{\partial G_{\gamma}}{\partial x_1^{\gamma}}
\end{align}
Eqns.~\ref{eqn:3chempot1a}~\&~\ref{eqn:3chempot1b} represent equality of
chemical potential of component 1 in pairs of phases $\alpha-\beta$ and
$\alpha-\gamma$. Similarly, solving
Eqns.~\ref{eqn:3el4},~\ref{eqn:3el5},~\&~\ref{eqn:3el6} for $\lambda_2$ gives

\begin{align}
  \lambda_2 &= \frac{\partial G_{\alpha}}{\partial x_2^{\alpha}}
             = \frac{\partial G_{\beta}}{\partial x_2^{\beta}}
             = \frac{\partial G_{\gamma}}{\partial x_2^{\gamma}}\\
  \label{eqn:3chempot2a}
  \therefore \frac{\partial G_{\alpha}}{\partial x_2^{\alpha}}
          &= \frac{\partial G_{\beta}}{\partial x_2^{\beta}}\\
  \label{eqn:3chempot2b}
             \frac{\partial G_{\alpha}}{\partial x_2^{\alpha}}
          &= \frac{\partial G_{\gamma}}{\partial x_2^{\gamma}}
\end{align}
Eqns.~\ref{eqn:3chempot2a}~\&~\ref{eqn:3chempot2b} represent equality of
chemical potential of component 2 in pairs of phases $\alpha-\beta$ and
$\alpha-\gamma$. Solving Eqns.~\ref{eqn:3el7},~\ref{eqn:3el8}~\&~\ref{eqn:3el9}
for $\lambda_3$ gives

\begin{align}
  \lambda_3 &= G_{\alpha}
             - \frac{\partial G_{\alpha}}{\partial x_1^{\alpha}} x_1^{\alpha}
             - \frac{\partial G_{\alpha}}{\partial x_2^{\alpha}} x_2^{\alpha}
             = G_{\beta}
             - \frac{\partial G_{\alpha}}{\partial x_1^{\alpha}} x_1^{\beta}
             - \frac{\partial G_{\alpha}}{\partial x_2^{\alpha}} x_2^{\beta}
             = G_{\gamma}
             - \frac{\partial G_{\alpha}}{\partial x_1^{\alpha}} x_1^{\gamma}
             - \frac{\partial G_{\alpha}}{\partial x_2^{\alpha}} x_2^{\gamma}\\
  \label{eqn:3grandpot1}
  \therefore G_{\beta} &= G_{\alpha}
             - \frac{\partial G_{\alpha}}{\partial x_1^{\alpha}}(x_1^{\alpha} - x_1^{\beta})
             - \frac{\partial G_{\alpha}}{\partial x_2^{\alpha}}(x_2^{\alpha} - x_2^{\beta})\\
  \label{eqn:3grandpot2}
  G_{\gamma} &= G_{\alpha}
              - \frac{\partial G_{\alpha}}{\partial x_1^{\alpha}}(x_1^{\alpha} - x_1^{\gamma})
              - \frac{\partial G_{\alpha}}{\partial x_2^{\alpha}}(x_2^{\alpha} - x_2^{\gamma})
\end{align}
These represent equality of the grand potentials between pairs of phases $\alpha-\beta$ and
$\alpha-\gamma$.

The system of Eqns.~\ref{eqn:3chempot1a}, \ref{eqn:3chempot1b},
\ref{eqn:3chempot2a}, \ref{eqn:3chempot2b}, \ref{eqn:3grandpot1} \&
\ref{eqn:3grandpot2} represents 6 equations for 6 unknowns, which can be solved
for the equilibrium plane containing the points
$G_{\alpha}(x_1^{\alpha},x_2^{\alpha}), G_{\beta}(x_1^{\beta},x_2^{\beta})$ and
$G_{\gamma}(x_1^{\gamma},x_2^{\gamma})$.

\section*{Put it All Together}

The code accompanying this document generates the equilibrium isothermobaric
(constant $T$ and $P$) phase diagram for a ternary alloy with three phases. The
program iterates through system compositions $(x_1, x_2)$ in the Gibbs simplex.
At each test point, test energies are computed for
\begin{enumerate}
  \item $G_{\alpha}(x_1, x_2)$
  \item $G_{\beta}(x_1, x_2)$
  \item $G_{\gamma}(x_1, x_2)$
  \item $G_{\alpha\beta} = v_{\alpha}G_{\alpha}(x_1^{\alpha}, x_2^{\alpha})
                         + v_{\beta}G_{\beta}(x_1^{\beta}, x_2^{\beta})$
  \item $G_{\alpha\gamma} = v_{\alpha}G_{\alpha}(x_1^{\alpha}, x_2^{\alpha})
                          + v_{\gamma}G_{\gamma}$
  \item $G_{\beta\gamma} = v_{\beta}G_{\beta}(x_1^{\beta}, x_2^{\beta})
                         + v_{\gamma}G_{\gamma}(x_1^{\gamma}, x_2^{\gamma})$
  \item $G_{\alpha\beta\gamma} = v_{\alpha}G_{\alpha}(x_1^{\alpha}, x_2^{\alpha})
                               + v_{\beta}G_{\beta}(x_1^{\beta}, x_2^{\beta})
                               + v_{\gamma}G_{\gamma}(x_1^{\gamma}, x_2^{\gamma})$
\end{enumerate}
The equilibrium phases and compositions corresponding to the system composition
$(x_1, x_2)$ are those corresponding to the minimum energy represented in that
set of 7 possibilities, subject to the constraint that all endpoints must lie
within the Gibbs simplex.
  
\end{document}
